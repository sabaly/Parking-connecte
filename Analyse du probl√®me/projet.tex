\documentclass{beamer}

\usepackage[french]{babel}
\usepackage[OT1]{fontenc}
\usepackage[utf8]{inputenc}


\usetheme[progressbar=frametitle]{metropolis}
\setbeamertemplate{frame numbering}[fraction]
\useoutertheme{metropolis}
\useinnertheme{metropolis}
\usefonttheme{metropolis}
\usecolortheme{spruce}
\setbeamercolor{background canvas}{bg=white}

\title{Gestionnaire de rendez-vous}
\subtitle{Agenda electronique : Système de gestion}
\institute{Transmission des Données et Sécurité de l'Information}
\date{}
\author{Ali Houssene Silahi\\Cheikh Tidiane Thiam\\Thierno Mamoudou Sabaly}

\begin{document}
\begin{frame}
	\titlepage
\end{frame}
\begin{frame}{Introduction}\vspace{3pt}
Ce chapitre du projet met l'accent sur des normes à respecter pour fixer un rendez-vous. Ainsi, notre reflexion portera sur les heures de rendez-vous, le nombre maximun de clients par jours....\\
sans plus tarder, explicitons tous cela un a un.
\end{frame}
\begin{frame}[t]{Idée générale}\vspace{3pt}
Les rendez-vous seront fixés en fonction de la disponibilité des medecins. Par conséquents, il nous faudra leurs plannings à tous. L'idée c'est que à chaque semaine, au plus tard le vendredi de la semaine, chaque medecin doit avoir communiquer son planning de la semaine suivante. Pour faire simple, il communiquera que les jours et heures auquels il sera prêt à recevoir des rendez-vous.
\end{frame}
\end{document}